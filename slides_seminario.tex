% Options for packages loaded elsewhere
\PassOptionsToPackage{unicode}{hyperref}
\PassOptionsToPackage{hyphens}{url}
%
\documentclass[
  ignorenonframetext,
]{beamer}
\usepackage{pgfpages}
\setbeamertemplate{caption}[numbered]
\setbeamertemplate{caption label separator}{: }
\setbeamercolor{caption name}{fg=normal text.fg}
\beamertemplatenavigationsymbolsempty
% Prevent slide breaks in the middle of a paragraph
\widowpenalties 1 10000
\raggedbottom
\setbeamertemplate{part page}{
  \centering
  \begin{beamercolorbox}[sep=16pt,center]{part title}
    \usebeamerfont{part title}\insertpart\par
  \end{beamercolorbox}
}
\setbeamertemplate{section page}{
  \centering
  \begin{beamercolorbox}[sep=12pt,center]{part title}
    \usebeamerfont{section title}\insertsection\par
  \end{beamercolorbox}
}
\setbeamertemplate{subsection page}{
  \centering
  \begin{beamercolorbox}[sep=8pt,center]{part title}
    \usebeamerfont{subsection title}\insertsubsection\par
  \end{beamercolorbox}
}
\AtBeginPart{
  \frame{\partpage}
}
\AtBeginSection{
  \ifbibliography
  \else
    \frame{\sectionpage}
  \fi
}
\AtBeginSubsection{
  \frame{\subsectionpage}
}
\usepackage{amsmath,amssymb}
\usepackage{lmodern}
\usepackage{iftex}
\ifPDFTeX
  \usepackage[T1]{fontenc}
  \usepackage[utf8]{inputenc}
  \usepackage{textcomp} % provide euro and other symbols
\else % if luatex or xetex
  \usepackage{unicode-math}
  \defaultfontfeatures{Scale=MatchLowercase}
  \defaultfontfeatures[\rmfamily]{Ligatures=TeX,Scale=1}
\fi
\usetheme[]{CambridgeUS}
% Use upquote if available, for straight quotes in verbatim environments
\IfFileExists{upquote.sty}{\usepackage{upquote}}{}
\IfFileExists{microtype.sty}{% use microtype if available
  \usepackage[]{microtype}
  \UseMicrotypeSet[protrusion]{basicmath} % disable protrusion for tt fonts
}{}
\makeatletter
\@ifundefined{KOMAClassName}{% if non-KOMA class
  \IfFileExists{parskip.sty}{%
    \usepackage{parskip}
  }{% else
    \setlength{\parindent}{0pt}
    \setlength{\parskip}{6pt plus 2pt minus 1pt}}
}{% if KOMA class
  \KOMAoptions{parskip=half}}
\makeatother
\usepackage{xcolor}
\newif\ifbibliography
\setlength{\emergencystretch}{3em} % prevent overfull lines
\providecommand{\tightlist}{%
  \setlength{\itemsep}{0pt}\setlength{\parskip}{0pt}}
\setcounter{secnumdepth}{-\maxdimen} % remove section numbering
\ifLuaTeX
  \usepackage{selnolig}  % disable illegal ligatures
\fi
\IfFileExists{bookmark.sty}{\usepackage{bookmark}}{\usepackage{hyperref}}
\IfFileExists{xurl.sty}{\usepackage{xurl}}{} % add URL line breaks if available
\urlstyle{same} % disable monospaced font for URLs
\hypersetup{
  pdftitle={Primeiros passos para criação de livros usando o pacote bookdown do R},
  pdfauthor={Cristian Villegas (clobos@usp.br)},
  hidelinks,
  pdfcreator={LaTeX via pandoc}}

\title{Primeiros passos para criação de livros usando o pacote bookdown
do R}
\author{Cristian Villegas
(\href{mailto:clobos@usp.br}{\nolinkurl{clobos@usp.br}})}
\date{03-05-2023}
\institute{USP, Departamento de Ciências Exatas}

\begin{document}
\frame{\titlepage}

\hypertarget{steps-for-publishing-your-own-bookdown-document}{%
\section{13 steps for publishing your own bookdown
document}\label{steps-for-publishing-your-own-bookdown-document}}

\begin{frame}{13 steps for publishing your own bookdown document}
\protect\hypertarget{steps-for-publishing-your-own-bookdown-document-1}{}
\begin{itemize}
\tightlist
\item
  Install bookdown R package (Yihui (2015, 2023)):
  \textbf{install.packages(``bookdown'')}
\item
  Create a GitHub account \url{https://github.com/}
\item
  Signing up in your GitHub account
\item
  Create a new repository in your GitHub called
  \textbf{my\_first\_bookdown}
\item
  Create a new project in RStudio and \textbf{Book project using
  bookdown}
\item
  Modify the \_bookdown.yml file
\item
  Create the util.R file
\item
  Build your first bookdown (Ctrl + Shift + B)
\item
  Add your files in your Github \textbf{my\_first\_bookdown} folder
\item
  Click Settings, Pages, Branch, None (Main), /root (/docs) and Save
\item
  Click \textbf{my\_first\_bookdown}
\item
  Wait for the green ball to appear
\item
  Click \textbf{\url{https://username.github.io/my_first_bookdown}}
\end{itemize}
\end{frame}

\hypertarget{modify-the-_bookdown.yml-file}{%
\section{Modify the \_bookdown.yml
file}\label{modify-the-_bookdown.yml-file}}

\begin{frame}[fragile]{Modify the \_bookdown.yml file}
\protect\hypertarget{modify-the-_bookdown.yml-file-1}{}
\begin{verbatim}
output_dir: docs
delete_merged_file: true
new_session: true

language:
  label:
    fig: 'Figura '
    tab: 'Tabela '

ui:
  chapter_name: 'Capítulo '
  
before_chapter_script: "util.R"

book_filename: "PDF_seminario" 
\end{verbatim}
\end{frame}

\hypertarget{add-the-util.r-file}{%
\section{Add the util.R file}\label{add-the-util.r-file}}

\begin{frame}[fragile]{Add the util.R file}
\protect\hypertarget{add-the-util.r-file-1}{}
\begin{verbatim}
knitr::opts_chunk$set(message = FALSE, 
                      warning = FALSE)
\end{verbatim}
\end{frame}

\hypertarget{exercise}{%
\section{Exercise}\label{exercise}}

\begin{frame}{Exercise}
\protect\hypertarget{exercise-1}{}
\textbf{Modify everything in your new bookdown}
\end{frame}

\hypertarget{more-details}{%
\section{More details}\label{more-details}}

\begin{frame}{More details}
\protect\hypertarget{more-details-1}{}
\begin{itemize}
\tightlist
\item
  \url{https://github.com/rstudio/bookdown-demo}
\item
  \url{https://bookdown.org/yihui/bookdown/}
\end{itemize}
\end{frame}

\hypertarget{references}{%
\section{References}\label{references}}

\begin{frame}{References}
\protect\hypertarget{references-1}{}
\begin{itemize}
\item
  Xie, Yihui (2015). Dynamic Documents with R and Knitr. 2nd ed.~Boca
  Raton, Florida: Chapman; Hall/CRC. \url{http://yihui.org/knitr/}.
\item
  Xie, Yihui (2023). Bookdown: Authoring Books and Technical Documents
  with r Markdown. \url{https://CRAN.R-project.org/package=bookdown}.
\end{itemize}
\end{frame}

\end{document}
